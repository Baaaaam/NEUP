
\subsection{Robustness assessment}

The second part of this project will be dedicated to the measurement of the
robustness to different modeling choices of different output metrics as a
function of the studied scenario.  In the early age of the nuclear fuel cycle
simulation because of the limitation of the computational power, the
capabilities of the fuel cycle simulation were limited and the subject to strong
simplification. But the refinement of the different models did grows with the
available computers power. 

The different actors of the fuel cycle community do not agree on the requirement
to use the highest modeling fidelity. While the primary intuition supports
always using the best modeling options, some experts argue that the different
hypothesis and unknown of a prospective scenario are largely dominate the
uncertainty on the output metrics; considering better modeling option is then a
waste of time and effort.

Nevertheless it is possible to pictures use cases where some higher level of
refinement are required to achieve correct or better conclusion. For example, a
simple description of the enrichment facility (as a simple relation between the
different assay with a SWU constrain) will be enough for the simulation of the
historical a national fleet, a fuel cycle evaluation of the JCPOA agreement will
require a fine description of the enrichment cascade\dots Moreover there is no
scientific evaluation of the importance on the output metrics uncertainty of the
modeling precision over the input errors and uncertainties. This part of the
project aims to evaluate the impact of the different modeling choices on the
output metrics and compare them to the one of the input metrics uncertainties.

The robustness part of the project will be divided into 3 phases, the experiment
definitions, the experiment building, and the analysis of the experiments.


\subsubsection{Experiment Definition}

As fuel cycle simulation can be used to support many different types of
decisions (facility siting, technology transitions, impacts of international
trade agreements, etc), a set of different nuclear fuel cycle scenarios will be
identified.

Multiple consultation meetings with the different stakeholders: politics,
industry, fco analyst and researchers. Those consultations will help to identify
the types of decisions that could be supported by different fuel cycle
calculation and the list of relevant output metrics for each applications. 
Some examples include: availability/reduction of resources (thorium vs uranium
vs plutonium), reduction of long term waste radio-toxicity, and minimization of
human manipulation of activated material.

In addition to the scenarios identified with those consultations, other nuclear
fuel cycle applications will be considered, some examples includes: US (large
heterogeneous fleet) and/or French (fleet undergoing transition to plutonium
recycling) historical fleet, EG-like transition as well as, non-proliferation
application such as Iran case.

In this compilation of use scenarios, about five will be identified as ``most
representative`` of the different nuclear fuel cycle usages and used as
''reference scenarios`` for the rest of this study. In addition to those
''reference scenarios'', set of output metrics will be identified for each use
cases, with an associated acceptable tolerances, within decisions/conclusions
can be made.


\subsubsection{The experiments}

This second task will be be split into two subtasks dedicated to robustness
assessment, the first one to variations in modeling choices, the second one to
input variations.  

Each identified scenario will be modeled with a range of existing Cyclus
facility models.  Previously funded efforts have resulted in a variety of
reactor models: fleet-based recipe reactors [4], individual recipe reactors [5],
and depletion-aware/capable reactors [6, 7, 8].  As the reactors become more
sophisticated, it may also be necessary to invoke more sophisticated models for
fuel fabrication that can respond to varying reactor fueling needs [7, 8]. The
Cyclus framework makes it possible to shuffle the various modeling options, for
example using a single depletion aware reactor with any fuel fabrication model,
to explore the impact of a single modeling change while the rest of the
simulation remains unchanged.  

In addition to probing the impact of modeling choice, this work will also seek
to understand how robust a given decision is to variations in the input
assumptions for a given scenario.  A second round of analyses will be performed
to establish the robustness of the output metrics to variation on the fuel cycle
parameters such as cooling/reprocessing time, enrichment tails, facility
throughput, as well as transition schedule (start date, speed\dots).

Combined, those analyses will provide a way to rank different modeling choices
and input variables by importance depending on the outcome of the fuel cycle
simulations.


\subsubsection{Experiment Analysis}

The final task will be dedicated to the analysis and the compilation of the
required modeling fidelity as well as the input parameters constrains required
to successfully support the different decision making processes.

Those analysis will provide guidance not only to the nuclear fuel cycle analysts
regarding the necessary level of model fidelity that is required to support
particular kinds of decision making, but also to fuel cycle researcher regarding
the improvement needed on the differents modeling capabilities.
