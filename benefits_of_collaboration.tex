\documentclass[dvips,12pt]{article}

% Any percent sign marks a comment to the end of the line

% Every latex document starts with a documentclass declaration like this
% The option dvips allows for graphics, 12pt is the font size, and article
%   is the style

\usepackage[letterpaper, margin=1in]{geometry}

\usepackage[pdftex]{graphicx}
\usepackage{url}
\usepackage[pdftex]{xcolor}
\usepackage{amsmath}
\usepackage{enumitem}
\usepackage{tabularx}

\newcommand{\ID}{15645}
\newcommand{\mytitle}{Developing an Approach to Verification and Validation\\ for Nuclear Fuel Cycle Simulation Tools}
\newcommand{\workscope}{MS-FC-1}
\newcommand{\timeframe}{10/01/2018 - 09/30/2021}


\usepackage{fancyhdr}
\pagestyle{fancy}
\fancyhf{}
\lfoot{2018 CFA Benefits of Collaboration \ID }
\rfoot{Page \thepage\ of \pageref{LastPage}}

\renewcommand{\headrulewidth}{0pt}
\renewcommand{\footrulewidth}{0.5pt}

\newcommand{\unc}[1]
{ \delta #1 }

\newcommand{\uncsq}[1]
{ \left(\unc{#1}\right)^2 }

\newcommand{\uncratio}[1]
{ \left(\frac{\unc{#1}}{#1}\right) }

\newcommand{\uncratiosq}[1]
{ \uncratio{#1}^2 }

\newcommand{\uncvector}[1]
{ \left[ \begin{array}{c} #1 \\ \delta #1 \end{array} \right] }

\newcommand{\comment}[1]
{{\bfseries \color{red} #1}}

\makeatletter
\renewcommand\section{\@startsection {section}{1}{\z@}%
                                   {-2.0ex \@plus -1ex \@minus -.2ex}%
                                   {2.3ex \@plus.2ex}%
                                   {\normalfont\bfseries}}% from \Large
\renewcommand\subsection{\@startsection{subsection}{2}{\z@}%
                                     {-2.0ex\@plus -1ex \@minus -.2ex}%
                                     {1.5ex \@plus .2ex}%
                                     {\normalfont\bfseries}}% from \large
\renewcommand\subsubsection{\@startsection{subsubsection}{3}{\z@}%
                                     {-2.0ex\@plus -1ex \@minus -.2ex}%
                                     {1.5ex \@plus .2ex}%
                                     {\normalfont\bfseries}}% from \normalsize
\makeatother


%----------------------------------------------------------------------------------------
%	DOCUMENT INFORMATION
%----------------------------------------------------------------------------------------
\begin{document}
\begin{centering}
  Benefits of Collaboration for DE-FOA-0001772\\
  \textbf{\large \mytitle}\\
  2018 CFA Benefits of Collaboration \ID\\
\end{centering}
\vspace{1em}

\noindent\textbf{Technical Workscope Identifier:} \workscope\\
\textbf{Principal Investigator:} Paul P.H. Wilson, Professor, University of Wisconsin-Madison\\
\textbf{Time Frame:} \timeframe\\

Dr.\ Paul Wilson received his Dr.-Ing. from the Technical University of
Karlsruhe, Germany, in 1998, and his PhD from the University of
Wisconsin-Madison in 1999. He joined the Engineering Physics Department in
2001 and founded the Computational Nuclear Engineering Research Group (CNERG)
where he directs research that delivers novel analysis capability for nuclear
systems. In addition to leading the effort to develop CAD-based neutronics
analysis capability across application areas, he has been involved in various
aspects of nuclear fuel cycle analysis. Most recently, Dr. Wilson was the
principal investigator in a pair of NEUP projects to develop the next
generation fuel cycle simulator, Cyclus, that is at the foundation of this
proposal.  Prior to that, Dr.\ Wilson led a NERI project to investigate the
economic impacts of repository benefits due to recycling spent nuclear fuel.
Some models from this work were incorporated into the VISION fuel cycle
simulator and investigated for the DANESS fuel cycle simulator.  This was
followed by leading the development of GENIUSv2, the fuel cycle simulator
element of the SINEMA effort funded by Idaho National Laboratory and Argonne
National Laboratory.  Dr. Wilson was a consultant for the Blue Ribbon
Commission on America’s Nuclear Future on the area of fuel cycle systems
analysis and fuel cycle simulators.



\label{LastPage}
\end{document}
