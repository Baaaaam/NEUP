
\subsection{\gls{FCCI}}

This task will build a collaborative effort to specify problems that span the
decision space of nuclear fuel cycle analysis, and develop tools that
facilitate the cataloging and comparison of results that arise from simulating
these problem specifications.  An important feature of the framework that
results is that it be agnostic to any particular \gls{NFCS} tool.  The
collaborative nature of this effort, including review from many parties, will
help ensure this.  An initial set of problems will be used to develop and test
the mechanisms for collaborative development, with more complex problems
emerging once those mechanisms have been demonstrated.  Any team can then use
these problem specifications to gain a deeper understanding of the
implications of different development/modeling choices that are embedded in
their tool of choice.

Participation will be solicited from an international community of developers
and users of \gls{NFCS} tools.  Given the number and national variety of these
partners, direct funding of their participation is beyond the scope of this
project.  However, sincere expressions of interest from many parties have
emerged during two recent workshops \cite{twofc.2015, twofc.2016} focused on this specific community.  Early
successes will be used to showcase the effort at conferences and workshops to
promote broader participation.

%% establish a framework with collaboration tools  for
%% defining comparison problems in a way that is agnostic to the particular \gls{NFCS}
%% tool being testing and that facilitates comparison of the results in a
%% constructive and revealing manner. This will be performed  by defining an
%% initial set of problems that any team can use to test their own \gls{NFCS} tool and
%% gain a deeper understanding of the implications of different
%% development/modeling choices that are embedded in that tool. It will be carried
%% out in collaboration with the broader \gls{NFCS} community. The first set of complete
%% analysis will be used as a showcase (through conferences contributions and
%% scientific publications) to promote the \gls{FCCI} activities among the fuel cycle
%% community. 


\subsubsection{Collaborative Framework for Problem Definition and Analysis} 

While it can be straightforward to define a problem for any single \gls{NFCS} tool,
different modeling approaches and assumptions across a set of tools makes it
difficult to define those problems in a sufficiently generic manner.
Furthermore, the set of output data produced by each tool is different both in
the abstract (which data is generated) and the concrete (the format of that
data). The success of an effort such as this lies in the ability for new
participants to easily access the problem definition and then also easily
compare their results to the previously generated and submitted results.

This sub-task will focus on developing and formalizing the infrastructure
needed for effective collaboration, including a generalized ontology for
defining canonical problem inputs and outputs, default relationships between
different fidelity of analysis, mechanisms for proposing and refining new
problem definitions, submission of output data sets from a particular tool,
and web-based services to facilitate and automate the non-judgmental
comparison of the submitted outputs\cite{scopatz.NT.2016}, all within
an online community.

The infrastructure for sharing canonically defined problems and their
solutions will result in a web-based set of capabilities, built as much as
possible upon already-available web services.  Services such as
Github\cite{github} and CircleCI\cite{circleci} are available to deliver
content, track changes in problem specifications, log contributions from
collaborators, and automatically (re)generate comparisons between contributed
output data sets.


\subsubsection{Problem definition} 

This sub-task will be focused on the definition of new canonical problems that are
useful for demonstrating fundamental behavior of \gls{NFCS} tools and/or
highlighting behaviors that distinguish different methodologies from each
other. A canonical definition of the different problems and exercises should
avoid any bias related to the usage of a specific simulation tool and allow
theoretically any tools (with any conceptual design) to participate by
simulating the same problem. The ontology developed above will help to
accomplish this goal. This will require working with the \gls{FCCI}
collaboration to synthesize the different modeling verification needs, and
conceptualize problems that satisfy these needs while allowing for deep
analysis of potential differences due to modeling choices.  Experience from
prior ad-hoc comparison efforts suggest that each problem’s definition may
require several updates based on the feedback of the different analysts as
their modeling progresses: reaching a canonical definition as an asymptotic
process.

\subsubsection{Fuel Cycle Simulations and Analysis} 

This sub-task will be dedicated to modeling the different problems of the
\gls{FCCI}, from unit tests to the sensitivity analysis, using Cyclus
\gls{NFCS}\cite{CYCLUS}. The results of these simulations will be submitted,
using the infrastructure developed in sub-task 1.1, for analysis and comparison
to submissions generated using other \gls{NFCS} tools. These results will
also, therefore, be available to other participants as part of the comparison
set for future submissions. The Cyclus framework includes multiple facility
packages of different fidelity levels, and the ability to swap individual
facility models within an otherwise constant fuel cycle scenario. Thus, Cyclus
can contribute multiple submissions for each problem in the \gls{FCCI}
framework, each using a different degree of fidelity. Therefore, the
\gls{FCCI} framework will also serve as a way to assess the impacts of these
different fidelities with the Cyclus ecosystem. As new problems are identified
and defined in sub-task 1.2, they will be completed under this sub-task.


