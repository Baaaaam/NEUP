\section{Schedule}

This project has two main tasks that, while conceptually related, do not
require their schedules to be closely coordinated.  Some of the capability
that is developed in the first task will be useful for conducting the second
task, but it does not strictly depend on it.


\begin{table}[!]
  \begin{center}
    \begin{tabular}{|c|p{0.42\textwidth}|c|p{0.3\textwidth}|}\hline\hline
      \textbf{Task} & \textbf{Milestone} & \textbf{Date$^*$} & \textbf{Deliverable} \\\hline\hline
      1.1 & \gls{FCCI} infrastructure - alpha &   & \multirow{3}{0.3\textwidth}{Report \& presentation at TWoFCS '19} \\
      1.2 & Initial \gls{FCCI} problem specifications & 9 & \\
      1.3 & Initial \gls{FCCI} problem analysis &  & \\\hline
      1.1 & \gls{FCCI} infrastructure - beta &   & \multirow{3}{0.3\textwidth}{Report \& presentation at TWoFCS '20} \\
      1.2 & Intermediate \gls{FCCI} problem specifications & 21 & \\
      1.3 & Intermediate \gls{FCCI} problem analysis &  & \\\hline
      1.1 & \gls{FCCI} infrastructure - release &  & \multirow{3}{0.3\textwidth}{Report \& presentation at TWoFCS '21} \\
      1.2 & Final \gls{FCCI} problem specifications & 33 & \\
      1.3 & Final \gls{FCCI} problem analysis &  & \\\hline
      1 & Final report on \gls{FCCI} & 36 & Final report\\
      \hline\hline
      2.1 & Report on stakeholder consultations & 9 & \multirow{1}{0.3\textwidth}{Report \& presentation at TWoFCS '19} \\
          & & &\\
      2.1 & Definintion of robustness scenarios & 12 & Annual Report\\
      2.2 & Analysis of model variation & 24 & Annual Report\\
      2.3 & Analysis of input variations & 30 & Presentation at TWoFCS '21\\\hline
      2 & Guidance on appropriate model fidelity and input requirements & 36 & Final report\\
      \hline\hline
    \end{tabular}\\
    $^*$ Months from start of project
  \end{center}
\end{table}

As the goals of this proposal are to solicit collaboration from many parties
in the \gls{NFCS} community, the annual Technical Workshops on Fuel Cycle
Simulation (TWoFCS), held each July, will be important opportunities for
reporting successes and growing the size of the collaboration.  If
notification of the award is sufficiently early, discussion with potential
collaborators will begin at TWoFCS '18, scheduled for July 10-11, 2018.

\section{Milestones \& Deliverables}

\subsection{Task 1: \gls{FCCI}}

The deliverables for this task consist of three coordinated parts: (a) the
infrastructure for specifying problems, sharing those specficiations and their
results, (b) a set of problem/scenario specifications, and (c) a database of
solutions contributed from many partners.  While (b) and (c) can exist
independent of (a) they will be shared with the \gls{NFCS} community in a
coordinated fashion.  An initial set of demonstration problems and their
solutions will be released in concert with the \emph{alpha} release of the
infrastructure after 9 months.  An intermediate set of problems and solutions,
expanded in scope and/or depth of analysis, will be released with the
\emph{beta} release of the infrastructure a year later (21 months).  The
``final'' set of problems and solutions, again expanded in breadth and/or
depth, will be released with v1.0 of the infrastructure after 33 months.  This
would be the final set of problems developed under this project, but as an
active community collaboration, it is expected that additional problems may
ultimately be submitted after this project ends.

At each stage, each problem will be fully documented using the infrastructure
developed within this task.  Part of that infrastructure, therefore, is a
template that defines all the necessary parameters to define a problem and an
ontology for ensuring that the problem is defined universally.

\subsection{Task 2: Robustness Assessment}

The first deliverable from this task will be a report on the stakeholder
consultations, identifying the types of scenario of most interests and the
types of output metrics from those scenarios of highest importance in
subsequent decision making.  This will be followed quickly by the definition
of specific scenarios to use in the assessment of robustness.

