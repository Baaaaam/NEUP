\section{Proposed Scope}

A wide array of \gls{NFCS} tools\cite{some,nfcs,tools} are used to study a
variety of different types of problems, ranging from the impact of individual
reactors or fuel types in an existing nuclear fleet\cite{new-technology-study}
to the analysis of complete transitions to new advanced nuclear fuel
cycles\cite{example-transition-scenario}.  Unlike most other simulation
domains with nuclear energy applications, however, the notion of rigorous
validation is not viable.  Because \gls{NFCS} tools are designed to consider
speculative technologies in futures that incorporate socio-economic decision
making, there exists no experimental basis against which to benchmark them.
This leaves the \gls{NFCS} community with a challenge in establishing
confidence in any individual result from any indvidual tool.  This work will
address this challenge, aiming to provide a framework for developing
confidence in both existing \gls{NFCS} tools, and any novel tools that may be
introduced in the future.

Software \gls{VnV} are well established and independent processes that are
typically used to develop confidence in the models and algorithms that have
been implemented in a simulation tool.  Verification is the process that tests
whether or not each model has been implemented correctly, while validation is
the process that tests whether or not the chosen model reflects the reality.
The former process is largely a matter of software testing and comparison to
analytical solutions.  The latter typically relies on comparison to
experimental data.  Beyond just declaring a simulation tool as ``correct'',
the validation process is useful to define the bounds of applicability of a
given tool, recognizing that some simpler models may be sufficient to describe
some types of real world systems, but not others.  Given the degree of
uncertainty inherent in many of the data used to populate such analyses,
determining the limits of any given model or tool may be less defined by the
specific numerical answers than by the robustness of conclusions that are
drawn from those answer.  

While many aspects of software verification for \gls{NFCS} tools can be
achieved with a comprehensive application of modern software testing methods,
validation of the full specturm of capabilities is virtually impossible.  To
date, two approaches have been explored: comparison with historical
developments of specific national nuclear fleets, and comparisons across
multiple tools of specific prospective technology transitions.

Historical comparisons are best suited to demonstrate the ability of a
\gls{NFCS} tool to track material evolutions through a system of facilities
with well-defined deployment histories, and thus require a careful accounting
of all material flows in that system.  Whether due to issues of national
security or industrial strategy, comprehensive data about existing nuclear
fleets is not widely available.  There are examples of specific countries
using internally compiled information for validation of internally developed
tools\cite{COSI6 - Coquelet}, but this information is not generally available
as a community benchmark.

However, validation against historical records is not sufficient to develop
confidence in the primary use case of \gls{NFCS} tools: understanding the
impacts of new fuel and/or reactor concepts during multi-decade (or
multi-century) transitions in the future.  Such scenario studies attempt to
model technologies with varying \glspl{TRL} through futures with uncertain
socio-economic constraints.  The \gls{NFCS} community has engaged in a number
of comparison studies in attempts to address this validation deficit\cite{IAEA
  - Benchmark Study on Nuclear Fuel Cycle Transition Scenarios, MIT -
  Guerin,Standardized verification of fuel cycle modeling -
  B.Feng,some,previous,comparison,studies}. Some were designed expressly to
compare different \gls{NFCS} tools and others were comparisons carried out in
the course of seeking insight into specific scenarios of interest. While these
exercises have been valuable to participants and have contributed to the
development of their specific tools, they have not been effective at
establishing a basis for wide-spread validation of \gls{NFCS} tools, as it
common in other domains.

The initial problem specification is often better suited to modeling by one
tool than another.  Given the very high-dimensional decision space for
modeling future nuclear fuel cycles, different tools often focus on different
subsets of that decision space.  Therefore, any given problem specification
may full specify a scenario for one tool but be missing important information
for another tool.  In many cases, the problem is redefined iteratively to
cover a subset of the decision space that is largely common to all tools.  As
the scope of any problem is reduced, it provides a narrower benchmark.  After
a number of such comparisons, there maybe a patchwork of problems that may be
partly overlapping in some parts of the decision space and completely ignoring
other parts.  This limits the utility of the set of problems in two ways:
there are parts of decision space that are not validated, and little insight
is gained regarding the domain of applicability of any given tool or its
underlying models. \emph{This project will develop a set of problem
  specifications that is designed to cover the decision space and completely
  as possible, and to provide different levels of fidelity to probe the domain
  of applicability of \gls{NFCS} tools.}

Another shortcoming of previous comparison efforts is that they have not
published the raw results of their analyses to enable future comparisons by
new or updated tools.  The participants benefited at the time from access to
each other's data and the ability to discuss discrepancies, small and large,
and discuss their possible origins.  Summary reports are then written that
include plots and some tables, but may not include enough data for another
analyst to compare their own results.  Furthermore, while the problem
specification may be described in such reports, the nuance with which each
tool may treat different parts of the decision space may be lost.  This could
be mitigated by publishing the complete input file for each tool.  \emph{This
  project will publish complete input files and output files from each
  participating \gls{NFCS} tool for use by future analysts.}

Collaborative development of the various scenario specifications is important
to ensure that the set of problems spans a sufficient fraction of the decision
space, supports a breadth of modeling options, and covers types of problems of
interest to the entire community.  This collaboration will also contribute to
a rich and diverse set of data available for ongoing use by the community.

Together, this set of problem specifications, specific implementations and
consequent output data will provide a rich set of scenarios for use by current
and future fuel cycle analysts to explore the nuances of this kind of
simulation.  Importantly, the problems will exist as part of a progression,
with increasing scope, fidelity and/or complexity.  As different teams/tools
participate in scenarios throughout this progression, the degree of agreement
and/or disagreement between the results will help to shed light on the
robustness of the results to different degrees of modeling fidelity.  A
collective outcome will be indications of what fidelity of model is necessary
to reach robust conclusions for particular types of problems or conclusions.
This will play an important role in developing overall confidence in the
analysis of nuclear fuel cycle futures, both by establishing minimum
fidelities for certain types of problems and by mitigating undue confidence in
higher fidelity approaches when those approaches do not add robustness.

%% \gls{NFCS} tools are used to study a broad range of scenarios, 
%% from the impact of a specific fuel or reactor type within an 
%% existing nuclear fleet to the prospective analysis of a complete transition to
%% advanced nuclear fuel cycle technologies\cite{example-transition-scenario}.
%% Unlike many simulation domains,
%% rigorous validation against experimental results is not viable for this kind of
%% simulation tool that combines speculative technologies with  aspects of human
%% socio-political decision
%% making. Nevertheless, it is important to establish confidence in the quality of
%% the results that are generated, and understand how the degree to which that confidence depends
%% on the use case. The primary existing ways to develop confidence in any \gls{NFCS} tool
%% is to compare with other similar tools, or with simplified sets of historical
%% data. In the former case, the conclusion of such a comparison often focuses on the
%% differences between the specific tools, but lacks quantitative information
%% on the precision of any of the individual results. The latter case only allows validation of
%% existing concepts and has limited applicability to new fuel or reactor concepts
%% with varying \glspl{TRL}.

%% Software verification and validation are two independent processes that
%% check that the software behaves as designed, and 
%% check that its response is close to reality, respectively.  That is,
%% has the model been implemented correctly \emph{versus} has the correct
%% model been implemented.

%% While many aspects of software verification can be achieved with a comprehensive application of 
%% modern software testing methods, validation in nuclear fuel cycle
%% simulation is almost impossible. Indeed, \gls{NFCS} tools aims to simulate, the
%% historic or prospective, large nuclear fuel cycle of an entity (company, region,
%% country, \ldots). For security and/or industrial strategic reasons, only small slices 
%% of the data about existing nuclear fuel cycles is publicly available, if any.
%% Some \gls{NFCS} tools, such as COSI developed by the CEA \cite{COSI6 - Coquelet}, have
%% been validated on real data but rigorous validation is limited in the \gls{NFCS}
%% community. Indeed, 
%% validation on historical data does not cover a large spectrum of cases, as one of the
%% primary
%% applications of \gls{NFCS} tools is the simulation of prospective future scenario for
%% which there is no historical data to establish confidence.

%% Several inter-comparison studies between \gls{NFCS} tools have been
%% performed\cite{IAEA - Benchmark Study on Nuclear Fuel Cycle Transition Scenarios, MIT - Guerin,some,previous,comparison,studies}. Because
%% of their format and the wide variety of \gls{NFCS} tools, those studies had a
%% modest impact on the ongoing development of \gls{NFCS} tools, despite some very interesting
%% features and conclusions.  Those studies are generally comparison between large
%% and complex transition scenarios whose complexities make it difficult to gain
%% deep understanding of the \gls{NFCS} tools' working processes. Moreover, those studies are
%% often led by one \gls{NFCS} tool development/using team, driving the specification of
%% the benchmark such that it is not always well defined for other tools.
%% These issues tend to limit the conclusions of such study, and make them particularly
%% difficult to use as the basis for developing new algorithms and/or tools. Additionally,
%% when published, it is common that either the definition
%% of the problem or detailed representations of the solutions generated by the
%% different tools are not publicly
%% available. It is then challenging for analysts not involved in the original
%% study to compare their own \gls{NFCS} tool with those results.

%% Although not formally constructed as benchmark exercises, some simulation campaigns
%% can also produce results for comparison purposes.  In the \gls{USDOE}['s]
%% \gls{FCO} campaign,  multiple analysts
%% using different \gls{NFCS} tools to simulate the same transition scenarios. The degree of
%% freedom in the transition specification varies from one scenario to an other,
%% which impacts the degree of agreement between the different calculations, but
%% those campaigns generally allow a deep comparison on specific problem between
%% several \gls{NFCS} tools \cite{Standardized verification of fuel cycle modeling -
%% B.Feng}.  Even though such comparisons are very valuable for the different tools
%% involved in the campaign, the added value for outsiders is very limited, as the
%% detailed specifications and results of those calculations are not always
%% publicly available. Moreover these studies have a very narrow focus on a few
%% specific transitions, intrinsically limiting the scope of the cross-comparison between the
%% \gls{NFCS} tools.



%% There are a number of different tools used around the world to support decisions
%% on nuclear fuel cycle options\cite{many,different,tools}.
%% While some tools offer a rigorous treatment of
%% the physics, they often require an experienced analyst to compute the results.
%% The risk is to offer more modeling precisions than is warranted given the
%% uncertainty in the inputs and related socio-political constraints. It might lead
%% to an over-complicated simulation with a very complex analysis, and a false
%% sense of certainty in the results. Furthermore, tool development/design
%% philosophy can potentially influence simulation outcome of a fuel cycle
%% analysis.

%% The first part of this project will be dedicated to creating a framework that allows
%% fuel cycle analysts to understand how specific modeling choices or tool designs
%% affect the simulation outcomes. This framework will include a variety of
%% problems defined to analyse a set of features and their range of implementations
%% in the different \gls{NFCS} tools. It will allow a collaborative definition of
%% benchmark problems, the evaluation of simulation outputs for these problems, and
%% the cataloging of each participant’s contribution.

%% The second part of this project will focus on linking modeling capability and
%% real study needs, aiming to assess the impact of modeling precision across
%% different facilities in the nuclear fuel cycle in order to determine the
%% robustness of the conclusions to those variations in modeling precision.


