
\subsection{Robustness assessment}

The second task of this project will be dedicated to the measurement of the
robustness of different output metrics to different modeling choices as a
function of the studied scenario.  In the early age of nuclear fuel cycle
simulation, limitations of the computational infrastructure, both hardware and
software, resulted in limitations to the scope, complexity and fidelity of the
simulations.  Over the last decade, the software platforms for fuel cycle
simulation have improved to permit more complex scenarios and take advantage
of growing hardware capabilities.

Currently, there can be disagreement within the community on the necessary
fidelity to achieve a robust conclusion for some problems.  This is confounded
by the non-overlapping features of different tools such that direct
comparisons are difficult. As with many simulation communities, there is a
natural tendency to steadily improve the modeling fidelity.  However, some
experts argue that the uncertainty in the input data for a prospective
scenario results in output uncertainties that render increased fidelity
unnecessary, at best, and misleading, at worst.  If true, then the time and
effort for improving fidelity can be unwarranted in many cases.

Nevertheless it is possible to describe use cases where some higher level of
refinement is required to achieve correct or adequate conclusions. For
example, a simple description of the enrichment facility (as a simple relation
between the different assay with a SWU constrain) will be enough for the
simulation of the history of a national fleet, but a fuel cycle evaluation of
the JCPOA agreement\cite{jcpoa} will require a more detailed model of the enrichment
cascades. Moreover there is no scientific evaluation of the importance of the
modeling precision versus the input errors and uncertainties on the output
metrics uncertainty. This part of the project aims to evaluate the impact of
the different modeling choices on the output metrics and compare them to that
of the input metrics uncertainties.

After defining the scenarios that will form the basis of the experiments, two
related and parallel sub-tasks will investigate different types of variations.
The first one will test robustness for variations in modeling choice under
constant input parameters, and the second one for variations in the input
parameters under constant modeling choice.

\subsubsection{Experiment Definition}

As fuel cycle simulation can be used to support many different types of
decisions (facility siting, technology transitions, impacts of international
trade agreements, etc), a set of diverse nuclear fuel cycle scenarios will be
identified through multiple consultations with different stakeholders:
policy-makers, industry, and researchers.  Those consultations will help to
identify the types of decisions that could be supported by different fuel
cycle calculations and the list of relevant output metrics for each
application.  Some examples include: availability/consumption of resources
(thorium vs uranium vs plutonium), reduction of long term waste
radiotoxicity, and minimization of human manipulation of radioactive material.

In addition to the scenarios identified with those consultations, other
nuclear fuel cycle applications will be considered, including historical
records of US (large heterogeneous fleet) and/or French (fleet undergoing
transition to plutonium recycling)\cite{courtin.phd} commercial
nuclear fleets and EG-like transitions\cite{Bo.ANSW.2016}.

In this compilation of use scenarios, some of them will be identified as
``most representative'' of the different \gls{NFCS} use cases and used as
``reference scenarios'' for the rest of this study. Sets of output metrics
will also be identified, with associated acceptable tolerances, within which
decisions/conclusions can be made. Each reference scenario will be
investigated across all identified set of output metrics.

These problems will also become part of the list of scenarios available in the
\gls{FCCI} described in task 1.

\subsubsection{Analysis of Robustness to Modeling Choices}

Each identified scenario will be modeled with a range of existing Cyclus
facility models.  Previously efforts have resulted in a variety of reactor
models: fleet-based recipe reactors \cite{carlsen.NT.2016}, individual recipe
reactors \cite{cycamore.1.5.0}, and depletion-aware/capable reactors
\cite{cyborg, brightlite.2015, cyclass.2016}.  As the reactors become more
sophisticated, it may also be necessary to invoke more sophisticated models for
fuel fabrication that can respond to varying reactor fueling needs
\cite{brightlite.2015, cyclass.2016}. The Cyclus framework makes it possible to
shuffle the various modeling options, for example using a single depletion aware
reactor with any fuel fabrication model, to explore the impact of a single
modeling change while the rest of the simulation remains unchanged.

Based on the feedback from the stakeholder consultations, important metrics
for each scenario will be carefully analyzed to determine the degree to which
the modeling choice affected that metric.  Furthermore, it will be important
to determine whether the change in that metric, if any, would result in
substantially different conclusions and/or decisions based on that analysis.

\subsubsection{Analysis of Robustness to Input Parameters}

In addition to probing the impact of modeling choice, this work will also seek
to understand how robust a given decision is to variations in the input
assumptions for a given scenario.  A second round of analyses will be performed
to establish the robustness of the output metrics to variation on the fuel cycle
parameters such as cooling/reprocessing time, enrichment tails, facility
throughput, as well as transition schedule (start date, speed, \dots).

Combined, these analyses will provide a way to rank different modeling choices
and input variables by importance depending on the outcome of the fuel cycle
simulations.  In some cases, it will be possible to compare across variations
in both modeling choice and input parameters, to determine if there are
interactions between them.  That is, are there circumstances where the
modeling choice makes the simulation more, or less, sensitive to variations in
the input parameters?

These analyses will provide guidance not only to the nuclear fuel cycle analysts
regarding the necessary level of model fidelity that is required to support
particular kinds of decision making, but also to fuel cycle researcher regarding
the improvement needed on the different modeling capabilities.
