\section{Facilities}

This project is entirely computational; the bulk of the work can be completed
on desktop-class computers.  In some cases, there will be benefits to using
cloud computing resources to complete a large set of independent fuel cycle
analysis problems.  The lead institution has access to free private cloud
computing resources on their campus, and experience with leveraging low-cost
commercial cloud computing options.

One of the deliverables of this project is a web-based framework for
describing a set of benchmark-like problems and sharing the results of those
problems contributed by many participants.  Freely available web services will
be used to prototype this framework.  Such services are expected to be
sufficient for the long term management of this framework, but if they are
determined to be unsuitable during the course of this work, alternatives will
not be developed within the time frame of this work.

\section{Unique Challenges \& Mitigations}

The biggest challenge will be in recruiting a sufficiently large set of
collaborators to participate by performing the problems with the \gls{NFCS}
tools of their choice.  The benefit and value of this effort depends on it
being adopted at a large enough scale within the community to become a
\emph{de facto} standard for demonstrating the capabilities of a \gls{NFCS}
tool.  At previous TWoFCS workshops, the community has indicated strong
support for this effort, but will need to participate voluntarily as this
project cannot provide funds for participation.  This challenge will be
mitigated by active participation in the program committees of the future
TWoFCS meetings, thereby including dedicated sessions in those programs for
discussion of the results of benchmark problems that exist in the \gls{FCCI}
system, as well as an invited talk about the state of the infrastructure.

\section{Quality Assurance \& Software Development Process}

\textit{This project will comply with all Quality Assurance requirements, as
described on the NEUP website.}

This work will follow the Cyclus development model.  We will employ a variety
of modern software project management tools: revision control, automated
testing, automated and manual documentation, bug tracking. Related projects at
are already managed under source code revision control system (git) that
provides detailed tracking of all changes to the code base. A test suite has
been developed and new tests will be added for each new capability. This test
suite will be automatically executed with each proposed change in the revision
control system to identify any flaws that may be introduced. Note that since
any software developed under this project is not considered simulation software
it will not be subject to a traditional \gls{VnV} process.  There will,
however, be rigorous and regular testing, including continuous integration
tests to ensure it is meeting its functional requirements. Automated
documentation tools are used in the source code to create a detailed reference
for the interfaces and additional background documentation. Out-of-code
detailed reports and publications will supplement the information on each new
capability.  Finally, a bug tracking system (GitHub issues) is deployed to
help users and developers to understand known issues and to track their
resolution as a developer community. As new capability is added, it will come
under the same quality assurance practices as described here for the existing
capabilities.
