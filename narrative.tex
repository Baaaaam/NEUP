\documentclass[dvips,12pt]{article}

% Any percent sign marks a comment to the end of the line

% Every latex document starts with a documentclass declaration like this The
% option dvips allows for graphics, 12pt is the font size, and article is the
% style
\usepackage[letterpaper, margin=1in]{geometry}


\usepackage[pdftex]{graphicx}
\usepackage{url}
\usepackage[pdftex]{xcolor}
\usepackage{amsmath}
\usepackage{enumitem}
\usepackage{tabularx}
\usepackage{todonotes}

\usepackage[acronym]{glossaries}
\newacronym{NFCS}{NFCS}{nuclear fuel cycle simulator}
\newacronym{TRL}{TRL}{technology readiness level}
\newacronym{FCO}{FCO}{Fuel Cycle Options}
\newacronym{USDOE}{USDOE}{U.S. Department of Energy}
\newacronym{VnV}{V\&V}{verification and validation}
\newacronym{FCCI}{FCCI}{Fuel Cycle Confidence Initiative}
\newacronym{CHTC}{CHTC}{Center for High-Throughput Computing}
\newacronym{ACI}{ACI}{Advanced Computing Initiative}
\newacronym{OSG}{OSG}{Open Science Grid}
\newacronym{HTC}{HTC}{high-throughput computing}
\newacronym{HPC}{HPC}{high-performance computing}


\usepackage{fancyhdr}
\pagestyle{fancy}
\fancyhf{}
\lfoot{2018 CFA Narrative xxxxxxxx}
\rfoot{Page \thepage\ of \pageref{LastPage}}

\renewcommand{\headrulewidth}{0pt}
\renewcommand{\footrulewidth}{0.5pt}

\newcommand{\unc}[1]
{ \delta #1 }

\newcommand{\uncsq}[1]
{ \left(\unc{#1}\right)^2 }

\newcommand{\uncratio}[1]
{ \left(\frac{\unc{#1}}{#1}\right) }

\newcommand{\uncratiosq}[1]
{ \uncratio{#1}^2 }

\newcommand{\uncvector}[1]
{ \left[ \begin{array}{c} #1 \\ \delta #1 \end{array} \right] }

\newcommand{\comment}[1]
{{\bfseries \color{red} #1}}

\makeatletter
\renewcommand\section{\@startsection {section}{1}{\z@}%
                                   {-2.0ex \@plus -1ex \@minus -.2ex}%
                                   {2.3ex \@plus.2ex}%
                                   {\normalfont\bfseries}}% from \Large
\renewcommand\subsection{\@startsection{subsection}{2}{\z@}%
                                     {-2.0ex\@plus -1ex \@minus -.2ex}%
                                     {1.5ex \@plus .2ex}%
                                     {\normalfont\bfseries}}% from \large
\renewcommand\subsubsection{\@startsection{subsubsection}{3}{\z@}%
                                     {-2.0ex\@plus -1ex \@minus -.2ex}%
                                     {1.5ex \@plus .2ex}%
                                     {\normalfont\bfseries}}% from \normalsize
\makeatother


%----------------------------------------------------------------------------------------
%DOCUMENT INFORMATION
%----------------------------------------------------------------------------------------
\begin{document}
\begin{centering} Application Narrative for
  DE-FOA-0001772\\
  \textbf{\large Developing an Approach to Verification and Validation\\ for Nuclear Fuel Cycle Simulation Tools}\\
  CFA-18-15645 Narrative\\
\end{centering}
%\tableofcontents

\vspace{1em}

\noindent\textbf{Technical Workscope Identifier:}  \hspace{1.5in}
\textbf{Time Frame:}

% Some context and general goal
\section{Project Objectives}

The objective of this project is to develop an open and collaborative approach
for establishing increased confidence in \gls{NFCS} tools. This approach will
directly address verification by publishing jointly-developed problems with
reference solutions.  It will indirectly address validation through a growing
set of publicly available solutions to those problems contributed by
developers of such tools.  Together, this approach will allow the nuclear
fuel cycle community to develop a more nuanced understanding of the results
produced by their tools and, in so doing, be better prepared to express
confidence in those results when communicating results both within and without
that community.

This kind of comparison will also build knowledge and understanding about the
differences between the various algorithms and models used by the many fuel
cycle simulation tools. In particular, it is important to understand the
circumstances under which the choice of algorithm and/or model influences the
conclusions that one might draw from such a simulation. Such knowledge could
not only help fuel cycle analysts to perform more pertinent and relevant
studies, but ultimately to convince stakeholders to rely more on nuclear fuel
cycle analysis in the decision making processes relative to nuclear
technology.

The development of this capability in an open and collaborative fashion will
support innovation in this space, as it makes it possible for new entrants to
immediately begin testing their tools against established solutions without
waiting for a coordinated benchmarking effort.  Such coordinated efforts have
occurred in the past, but their results are generally not published in a way
that invites comparison by future analysts who were not part of the initial
exercise.


\section{Proposed Scope}

A wide array of \gls{NFCS} tools\cite{some,nfcs,tools} are used to study a
variety of different types of problems, ranging from the impact of individual
reactors or fuel types in an existing nuclear fleet\cite{new-technology-study}
to the analysis of complete transitions to new advanced nuclear fuel
cycles\cite{example-transition-scenario}.  Unlike most other simulation
domains with nuclear energy applications, however, the notion of rigorous
validation is not viable.  Because \gls{NFCS} tools are designed to consider
speculative technologies in futures that incorporate socio-economic decision
making, there exists no experimental basis against which to benchmark them.
This leaves the \gls{NFCS} community with a challenge in establishing
confidence in any individual result from any indvidual tool.  This work will
address this challenge, aiming to provide a framework for developing
confidence in both existing \gls{NFCS} tools, and any novel tools that may be
introduced in the future.

Software \gls{VnV} are well established and independent processes that are
typically used to develop confidence in the models and algorithms that have
been implemented in a simulation tool.  Verification is the process that tests
whether or not each model has been implemented correctly, while validation is
the process that tests whether or not the chosen model reflects the reality.
The former process is largely a matter of software testing and comparison to
analytical solutions.  The latter typically relies on comparison to
experimental data.  Beyond just declaring a simulation tool as ``correct'',
the validation process is useful to define the bounds of applicability of a
given tool, recognizing that some simpler models may be sufficient to describe
some types of real world systems, but not others.  Given the degree of
uncertainty inherent in many of the data used to populate such analyses,
determining the limits of any given model or tool may be less defined by the
specific numerical answers than by the robustness of conclusions that are
drawn from those answer.  

While many aspects of software verification for \gls{NFCS} tools can be
achieved with a comprehensive application of modern software testing methods,
validation of the full specturm of capabilities is virtually impossible.  To
date, two approaches have been explored: comparison with historical
developments of specific national nuclear fleets, and comparisons across
multiple tools of specific prospective technology transitions.

Historical comparisons are best suited to demonstrate the ability of a
\gls{NFCS} tool to track material evolutions through a system of facilities
with well-defined deployment histories, and thus require a careful accounting
of all material flows in that system.  Whether due to issues of national
security or industrial strategy, comprehensive data about existing nuclear
fleets is not widely available.  There are examples of specific countries
using internally compiled information for validation of internally developed
tools\cite{COSI6 - Coquelet}, but this information is not generally available
as a community benchmark.

However, validation against historical records is not sufficient to develop
confidence in the primary use case of \gls{NFCS} tools: understanding the
impacts of new fuel and/or reactor concepts during multi-decade (or
multi-century) transitions in the future.  Such scenario studies attempt to
model technologies with varying \glspl{TRL} through futures with uncertain
socio-economic constraints.  The \gls{NFCS} community has engaged in a number
of comparison studies in attempts to address this validation deficit\cite{IAEA
  - Benchmark Study on Nuclear Fuel Cycle Transition Scenarios, MIT -
  Guerin,Standardized verification of fuel cycle modeling -
  B.Feng,some,previous,comparison,studies}. Some were designed expressly to
compare different \gls{NFCS} tools and others were comparisons carried out in
the course of seeking insight into specific scenarios of interest. While these
exercises have been valuable to participants and have contributed to the
development of their specific tools, they have not been effective at
establishing a basis for wide-spread validation of \gls{NFCS} tools, as it
common in other domains.

The initial problem specification is often better suited to modeling by one
tool than another.  Given the very high-dimensional decision space for
modeling future nuclear fuel cycles, different tools often focus on different
subsets of that decision space.  Therefore, any given problem specification
may full specify a scenario for one tool but be missing important information
for another tool.  In many cases, the problem is redefined iteratively to
cover a subset of the decision space that is largely common to all tools.  As
the scope of any problem is reduced, it provides a narrower benchmark.  After
a number of such comparisons, there maybe a patchwork of problems that may be
partly overlapping in some parts of the decision space and completely ignoring
other parts.  This limits the utility of the set of problems in two ways:
there are parts of decision space that are not validated, and little insight
is gained regarding the domain of applicability of any given tool or its
underlying models. \emph{This project will develop a set of problem
  specifications that is designed to cover the decision space and completely
  as possible, and to provide different levels of fidelity to probe the domain
  of applicability of \gls{NFCS} tools.}

Another shortcoming of previous comparison efforts is that they have not
published the raw results of their analyses to enable future comparisons by
new or updated tools.  The participants benefited at the time from access to
each other's data and the ability to discuss discrepancies, small and large,
and discuss their possible origins.  Summary reports are then written that
include plots and some tables, but may not include enough data for another
analyst to compare their own results.  Furthermore, while the problem
specification may be described in such reports, the nuance with which each
tool may treat different parts of the decision space may be lost.  This could
be mitigated by publishing the complete input file for each tool.  \emph{This
  project will publish complete input files and output files from each
  participating \gls{NFCS} tool for use by future analysts.}

Collaborative development of the various scenario specifications is important
to ensure that the set of problems spans a sufficient fraction of the decision
space, supports a breadth of modeling options, and covers types of problems of
interest to the entire community.  This collaboration will also contribute to
a rich and diverse set of data available for ongoing use by the community.

Together, this set of problem specifications, specific implementations and
consequent output data will provide a rich set of scenarios for use by current
and future fuel cycle analysts to explore the nuances of this kind of
simulation.  Importantly, the problems will exist as part of a progression,
with increasing scope, fidelity and/or complexity.  As different teams/tools
participate in scenarios throughout this progression, the degree of agreement
and/or disagreement between the results will help to shed light on the
robustness of the results to different degrees of modeling fidelity.  A
collective outcome will be indications of what fidelity of model is necessary
to reach robust conclusions for particular types of problems or conclusions.
This will play an important role in developing overall confidence in the
analysis of nuclear fuel cycle futures, both by establishing minimum
fidelities for certain types of problems and by mitigating undue confidence in
higher fidelity approaches when those approaches do not add robustness.

%% \gls{NFCS} tools are used to study a broad range of scenarios, 
%% from the impact of a specific fuel or reactor type within an 
%% existing nuclear fleet to the prospective analysis of a complete transition to
%% advanced nuclear fuel cycle technologies\cite{example-transition-scenario}.
%% Unlike many simulation domains,
%% rigorous validation against experimental results is not viable for this kind of
%% simulation tool that combines speculative technologies with  aspects of human
%% socio-political decision
%% making. Nevertheless, it is important to establish confidence in the quality of
%% the results that are generated, and understand how the degree to which that confidence depends
%% on the use case. The primary existing ways to develop confidence in any \gls{NFCS} tool
%% is to compare with other similar tools, or with simplified sets of historical
%% data. In the former case, the conclusion of such a comparison often focuses on the
%% differences between the specific tools, but lacks quantitative information
%% on the precision of any of the individual results. The latter case only allows validation of
%% existing concepts and has limited applicability to new fuel or reactor concepts
%% with varying \glspl{TRL}.

%% Software verification and validation are two independent processes that
%% check that the software behaves as designed, and 
%% check that its response is close to reality, respectively.  That is,
%% has the model been implemented correctly \emph{versus} has the correct
%% model been implemented.

%% While many aspects of software verification can be achieved with a comprehensive application of 
%% modern software testing methods, validation in nuclear fuel cycle
%% simulation is almost impossible. Indeed, \gls{NFCS} tools aims to simulate, the
%% historic or prospective, large nuclear fuel cycle of an entity (company, region,
%% country, \ldots). For security and/or industrial strategic reasons, only small slices 
%% of the data about existing nuclear fuel cycles is publicly available, if any.
%% Some \gls{NFCS} tools, such as COSI developed by the CEA \cite{COSI6 - Coquelet}, have
%% been validated on real data but rigorous validation is limited in the \gls{NFCS}
%% community. Indeed, 
%% validation on historical data does not cover a large spectrum of cases, as one of the
%% primary
%% applications of \gls{NFCS} tools is the simulation of prospective future scenario for
%% which there is no historical data to establish confidence.

%% Several inter-comparison studies between \gls{NFCS} tools have been
%% performed\cite{IAEA - Benchmark Study on Nuclear Fuel Cycle Transition Scenarios, MIT - Guerin,some,previous,comparison,studies}. Because
%% of their format and the wide variety of \gls{NFCS} tools, those studies had a
%% modest impact on the ongoing development of \gls{NFCS} tools, despite some very interesting
%% features and conclusions.  Those studies are generally comparison between large
%% and complex transition scenarios whose complexities make it difficult to gain
%% deep understanding of the \gls{NFCS} tools' working processes. Moreover, those studies are
%% often led by one \gls{NFCS} tool development/using team, driving the specification of
%% the benchmark such that it is not always well defined for other tools.
%% These issues tend to limit the conclusions of such study, and make them particularly
%% difficult to use as the basis for developing new algorithms and/or tools. Additionally,
%% when published, it is common that either the definition
%% of the problem or detailed representations of the solutions generated by the
%% different tools are not publicly
%% available. It is then challenging for analysts not involved in the original
%% study to compare their own \gls{NFCS} tool with those results.

%% Although not formally constructed as benchmark exercises, some simulation campaigns
%% can also produce results for comparison purposes.  In the \gls{USDOE}['s]
%% \gls{FCO} campaign,  multiple analysts
%% using different \gls{NFCS} tools to simulate the same transition scenarios. The degree of
%% freedom in the transition specification varies from one scenario to an other,
%% which impacts the degree of agreement between the different calculations, but
%% those campaigns generally allow a deep comparison on specific problem between
%% several \gls{NFCS} tools \cite{Standardized verification of fuel cycle modeling -
%% B.Feng}.  Even though such comparisons are very valuable for the different tools
%% involved in the campaign, the added value for outsiders is very limited, as the
%% detailed specifications and results of those calculations are not always
%% publicly available. Moreover these studies have a very narrow focus on a few
%% specific transitions, intrinsically limiting the scope of the cross-comparison between the
%% \gls{NFCS} tools.



%% There are a number of different tools used around the world to support decisions
%% on nuclear fuel cycle options\cite{many,different,tools}.
%% While some tools offer a rigorous treatment of
%% the physics, they often require an experienced analyst to compute the results.
%% The risk is to offer more modeling precisions than is warranted given the
%% uncertainty in the inputs and related socio-political constraints. It might lead
%% to an over-complicated simulation with a very complex analysis, and a false
%% sense of certainty in the results. Furthermore, tool development/design
%% philosophy can potentially influence simulation outcome of a fuel cycle
%% analysis.

%% The first part of this project will be dedicated to creating a framework that allows
%% fuel cycle analysts to understand how specific modeling choices or tool designs
%% affect the simulation outcomes. This framework will include a variety of
%% problems defined to analyse a set of features and their range of implementations
%% in the different \gls{NFCS} tools. It will allow a collaborative definition of
%% benchmark problems, the evaluation of simulation outputs for these problems, and
%% the cataloging of each participant’s contribution.

%% The second part of this project will focus on linking modeling capability and
%% real study needs, aiming to assess the impact of modeling precision across
%% different facilities in the nuclear fuel cycle in order to determine the
%% robustness of the conclusions to those variations in modeling precision.




\section{Logical path to accomplishing scope}

This project has two primary tasks:
\begin{enumerate}
\item developing an open, collaborative benchmark framework allowing real
  inter-code comparison, and
\item using this framework to understand what modeling fidelity is required to
  ensure the robustness of conclusions for different fuel cycle scenarios.
\end{enumerate}


% Collaborative Benchmark

\subsection{\gls{FCCI}}

This task will build a collaborative effort to specify problems that span the
decision space of nuclear fuel cycle analysis, and develop tools that
facilitate the cataloging and comparison of results that arise from simulating
these problem specifications.  An important feature of the framework that
results is that it be agnostic to any particular \gls{NFCS} tool.  The
collaborative nature of this effort, including review from many parties, will
help ensure this.  An initial set of problems will be used to develop and test
the mechisms for collaborative development, with more complex problems
emerging once those mechanisms have been demonstrated.  Any team can then use
these problem specifications to gain a deeper understanding of the
implications of different development/modeling choices that are embedded in
their tool of choice.

Participation will be solicited from an international community of developers
and users of \gls{NFCS} tools.  Given the number and national variety of these
partners, direct funding of their participation is beyond the scope of this
project.  However, sincere expressions of interest from many parties have
emerged during two recent workshops focused on this specific community.  Early
successes will be used to showcase the effort at conferences and workshops to
promote broader participation.

%% establish a framework with collaboration tools  for
%% defining comparison problems in a way that is agnostic to the particular \gls{NFCS}
%% tool being testing and that facilitates comparison of the results in a
%% constructive and revealing manner. This will be performed  by defining an
%% initial set of problems that any team can use to test their own \gls{NFCS} tool and
%% gain a deeper understanding of the implications of different
%% development/modeling choices that are embedded in that tool. It will be carried
%% out in collaboration with the broader \gls{NFCS} community. The first set of complete
%% analysis will be used as a showcase (through conferences contributions and
%% scientific publications) to promote the \gls{FCCI} activities among the fuel cycle
%% community. 


\subsubsection{Collaborative Framework for Problem Definition and Analysis} 

While it can be straightforward to define a problem for any single \gls{NFCS} tool,
different modeling approaches and assumptions across a set of tools makes it
difficult to define those problems in a sufficiently generic manner.
Furthermore, the set of output data produced by each tool is different both in
the abstract (which data is generated) and the concrete (the format of that
data). The success of an effort such as this lies in the ability for new
participants to easily access the problem definition and then also easily
compare their results to the previously generated and submitted results.

This subtask will focus on developing and formalizing the infrastructure
needed for effective collaboration, including a generalized ontology for
defining canonical problem inputs and outputs, default relationships between
different fidelities of analysis, mechanisms for proposing and refining new
problem definitions, submission of output data sets from a particular tool,
and web-based services to facilitate and automate the non-judgmental
comparison of the submitted outputs[3], all within an online community.

\subsubsection{Problem definition} 

This subtask will be focused on the definition of new canonical problems that are
useful for demonstrating fundamental behavior of \gls{NFCS} tools and/or
highlighting behaviors that distinguish different methodologies from each
other. A canonical definition of the different problems and exercises should
avoid any bias related to the usage of a specific simulation tool and allow
theoretically any tools (with any conceptual design) to participate by
simulating the same problem. The ontology developed above will help to
accomplish this goal. This will require working with the \gls{FCCI}
collaboration to synthesize the different modeling verification needs, and
conceptualize problems that satisfy these needs while allowing for deep
analysis of potential differences due to modeling choices.  Experience from
prior ad-hoc comparison efforts suggest that each problem’s definition may
require several updates based on the feedback of the different analysts as
their modeling progresses: reaching a canonical definition as an asymptotic
process.

\subsubsection{Fuel Cycle Simulations and Analysis} 

This subtask will be dedicated to modeling the different problems of the
\gls{FCCI}, from unit tests to the sensitivity analysis, using the Cyclus
\gls{NFCS}[2] developed at the UW-Madison. The results of these simulations
will be submitted, using the infrastructure developed in subtask 1.1, for
analysis and comparison to submissions generated using other \gls{NFCS}
tools. These results will also, therefore, be available to other participants
as part of the comparison set for future submissions. The Cyclus framework
includes multiple facility packages of different fidelity levels, and the
ability to swap individual facility models within an otherwise constant fuel
cycle scenario. Thus, Cyclus can contribute multiple submissions for each
problem in the \gls{FCCI} framework, each using a different degree of
fidelity. Therefore, the \gls{FCCI} framework will also serve as a way to
assess the impacts of these different fidelities with the Cyclus ecosystem. As
new problems are identified and defined in subtask 1.2, they will be completed
under this subtask.




% Robustness part

\subsection{Robustness assessment}

The second task of this project will be dedicated to the measurement of the
robustness of different output metrics to different modeling choices as a
function of the studied scenario.  In the early age of nuclear fuel cycle
simulation, limitations of the computational infrastructure, both hardware and
software, resulted in limitations to the scope, complexity and fidelity of the
simulations.  Over the last decade, the software platforms for fuel cycle
simulation have improved to permit more complex scenarios and take advantage
of growing hardware capabilities.

Currently, there can be disagreement within the community on the necessary
fidelity to achieve a robust conclusion for some problems.  This is confounded
by the non-overlapping features of different tools such that direct
comparisons are difficult. As with many simulation communities, there is a
natural tendency to steadily improve the modeling fidelity.  However, some
experts argue that the uncertainty in the input data for a prospective
scenario result in output uncertainties that render increased fidelity
unnecessary, at best, and misleading, at worst.  If true, then the time and
effort for improving fidelity can be unwarranted in many cases.

Nevertheless it is possible to describe use cases where some higher level of
refinement is required to achieve correct or adequate conclusions. For
example, a simple description of the enrichment facility (as a simple relation
between the different assay with a SWU constrain) will be enough for the
simulation of the history of a national fleet, but a fuel cycle evaluation of
the JCPOA agreement will require a more detailed model of the enrichment
cascades. Moreover there is no scientific evaluation of the importance of the
modeling precision versus the input errors and uncertainties on the output
metrics uncertainty. This part of the project aims to evaluate the impact of
the different modeling choices on the output metrics and compare them to that
of the input metrics uncertainties.

The robustness part of the project will be divided into 3 phases, the experiment
definitions, the experiment building, and the analysis of the experiments.

\subsubsection{Experiment Definition}

As fuel cycle simulation can be used to support many different types of
decisions (facility siting, technology transitions, impacts of international
trade agreements, etc), a set of diverse nuclear fuel cycle scenarios will be
identified through multiple consultations with different stakeholders:
policy-makers, industry, and researchers.  Those consultations will help to
identify the types of decisions that could be supported by different fuel
cycle calculations and the list of relevant output metrics for each
application.  Some examples include: availability/consumption of resources
(thorium vs uranium vs plutonium), reduction of long term waste
radiotoxicity, and minimization of human manipulation of radioactive material.

In addition to the scenarios identified with those consultations, other
nuclear fuel cycle applications will be considered, including historical
records of US (large heterogeneous fleet) and/or French (fleet undergoing
transition to plutonium recycling)\cite{CLASS group work on this} commercial
nuclear fleets andd EG-like transition\cite{BoFeng}.

In this compilation of use scenarios, some of them will be identified as
``most representative'' of the different \gls{NFCS} use cases and used as
``reference scenarios'' for the rest of this study. Sets of output metrics
will also be identified, with associated acceptable tolerances, within which
decisions/conclusions can be made. Each reference scenario will be
investigated across all identified set of output metrics.

These problems will also become part of the list of scenarios available in the
\gls{FCCI} described in task 1.

\subsubsection{The experiments}

This second subtask will be further split into two components dedicated to
robustness assessment, the first one for variations in modeling choice under
constant input parameters, the second one for variations in the input
parameters under constant modeling choice.

Each identified scenario will be modeled with a range of existing Cyclus
facility models.  Previously efforts have resulted in a variety of reactor
models: fleet-based recipe reactors \cite{4}, individual recipe reactors
\cite{5}, and depletion-aware/capable reactors \cite{6, 7, 8}.  As the reactors
become more sophisticated, it may also be necessary to invoke more
sophisticated models for fuel fabrication that can respond to varying reactor
fueling needs \cite{7, 8}. The Cyclus framework makes it possible to shuffle the
various modeling options, for example using a single depletion aware reactor
with any fuel fabrication model, to explore the impact of a single modeling
change while the rest of the simulation remains unchanged.

In addition to probing the impact of modeling choice, this work will also seek
to understand how robust a given decision is to variations in the input
assumptions for a given scenario.  A second round of analyses will be performed
to establish the robustness of the output metrics to variation on the fuel cycle
parameters such as cooling/reprocessing time, enrichment tails, facility
throughput, as well as transition schedule (start date, speed, \dots).

Combined, these analyses will provide a way to rank different modeling choices
and input variables by importance depending on the outcome of the fuel cycle
simulations.

\subsubsection{Experiment Analysis}

The final subtask will be dedicated to the analysis and the compilation of the
required modeling fidelity as well as the input parameters constraints required
to successfully support the different decision making processes.

Those analysis will provide guidance not only to the nuclear fuel cycle analysts
regarding the necessary level of model fidelity that is required to support
particular kinds of decision making, but also to fuel cycle researcher regarding
the improvement needed on the differents modeling capabilities.



%\input{narrative_4_
% Timeline - Milestone - Deliverables 
\input{narrative_5}

\section{Quality Assurance \& Software Development Process}

\textit{This project will comply with all Quality Assurance requirements, as
described on the NEUP website.}

This work will follow the Cyclus development model.  We will employ a variety of
modern software project management tools: revision control, automated testing,
automated and manual documentation, bug tracking. Related projects at are
already managed under source code revision control system (git) that provides
detailed tracking of all changes to the code base. A test suite has been
developed and new tests will be added for each new capability. This test suite
will be automatically executed with each proposed change in the revision control
system to identify any flaws that may be introduced. Automated documentation
tools are used in the source code to create a detailed reference for the
interfaces and additional background documentation. Out-of-code detailed reports
and publications will supplement the information on each new capability.
Finally, a bug tracking system (GitHub issues) is deployed to help users and
developers to understand known issues and to track their resolution as a
developer community. As new capability is added, it will come under the same
quality assurance practices as described here for the existing capabilities.

%----------------------------------------------------------------------------------------
%BIBLIOGRAPHY
%----------------------------------------------------------------------------------------

\bibliographystyle{narrative}

\bibliography{narrative}

%----------------------------------------------------------------------------------------

\label{LastPage} \end{document}
