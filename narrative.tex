\documentclass[dvips,11pt]{article}

% Any percent sign marks a comment to the end of the line

% Every latex document starts with a documentclass declaration like this
% The option dvips allows for graphics, 12pt is the font size, and article
%   is the style

\usepackage[letterpaper, margin=1in]{geometry}

\usepackage[pdftex]{graphicx}
\usepackage{url}
\usepackage[pdftex]{xcolor}
\usepackage{amsmath}
\usepackage{enumitem}
\usepackage{tabularx}

\usepackage{fancyhdr}
\pagestyle{fancy}
\fancyhf{}
\lfoot{20XX CFA Narrative xxxxx}
\rfoot{Page \thepage\ of \pageref{LastPage}}

\renewcommand{\headrulewidth}{0pt}
\renewcommand{\footrulewidth}{0.5pt}

\newcommand{\unc}[1]
{ \delta #1 }

\newcommand{\uncsq}[1]
{ \left(\unc{#1}\right)^2 }

\newcommand{\uncratio}[1]
{ \left(\frac{\unc{#1}}{#1}\right) }

\newcommand{\uncratiosq}[1]
{ \uncratio{#1}^2 }

\newcommand{\uncvector}[1]
{ \left[ \begin{array}{c} #1 \\ \delta #1 \end{array} \right] }

\newcommand{\comment}[1]
{{\bfseries \color{red} #1}}

\makeatletter
\renewcommand\section{\@startsection {section}{1}{\z@}%
                                   {-2.0ex \@plus -1ex \@minus -.2ex}%
                                   {2.3ex \@plus.2ex}%
                                   {\normalfont\bfseries}}% from \Large
\renewcommand\subsection{\@startsection{subsection}{2}{\z@}%
                                     {-2.0ex\@plus -1ex \@minus -.2ex}%
                                     {1.5ex \@plus .2ex}%
                                     {\normalfont\bfseries}}% from \large
\renewcommand\subsubsection{\@startsection{subsubsection}{3}{\z@}%
                                     {-2.0ex\@plus -1ex \@minus -.2ex}%
                                     {1.5ex \@plus .2ex}%
                                     {\normalfont\bfseries}}% from \normalsize
\makeatother


%----------------------------------------------------------------------------------------
%	DOCUMENT INFORMATION
%----------------------------------------------------------------------------------------
\begin{document}
\begin{centering}
  Application Narrative for DE-FOA-XXXXXXXXXXX\\
  \textbf{\large }\\
  20XX CFA Narrative xxxxxxx\\
\end{centering}
\vspace{1em}

\noindent\textbf{Technical Workscope Identifier:} #TBD \hspace{1.5in}
\textbf{Time Frame:} 
\section{}

\section{Quality Assurance \& Software Development Process}

\textit{This project will comply with all Quality Assurance requirements, as
described on the NEUP website.}  

This work will follow the Cyclus development model.  We will employ a variety of
modern software project management tools: revision control, automated testing,
automated and manual documentation, bug tracking. Related projects at are
already managed under source code revision control system (git) that provides
detailed tracking of all changes to the code base. A test suite has been
developed and new tests will be added for each new capability. This test suite
will be automatically executed with each proposed change in the revision control
system to identify any flaws that may be introduced. Automated documentation
tools are used in the source code to create a detailed reference for the
interfaces and additional background documentation. Out-of-code detailed reports
and publications will supplement the information on each new capability.
Finally, a bug tracking system (GitHub issues) is deployed to help users and
developers to understand known issues and to track their resolution as a
developer community. As new capability is added, it will come under the same
quality assurance practices as described here for the existing capabilities.

%----------------------------------------------------------------------------------------
%	BIBLIOGRAPHY
%----------------------------------------------------------------------------------------

\bibliographystyle{narrative}

\bibliography{narrative}

%----------------------------------------------------------------------------------------

\label{LastPage}
\end{document}
