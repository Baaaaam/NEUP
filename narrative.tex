\documentclass[dvips,11pt]{article}

% Any percent sign marks a comment to the end of the line

% Every latex document starts with a documentclass declaration like this The
% option dvips allows for graphics, 12pt is the font size, and article is the
% style
\usepackage[letterpaper, margin=1in]{geometry}


\usepackage[pdftex]{graphicx}
\usepackage{url}
\usepackage[pdftex]{xcolor}
\usepackage{amsmath}
\usepackage{enumitem}
\usepackage{tabularx}

\usepackage{fancyhdr}
\pagestyle{fancy}
\fancyhf{}
\lfoot{20XX CFA Narrative xxxxxxxx}
\rfoot{Page \thepage\ of \pageref{LastPage}}

\renewcommand{\headrulewidth}{0pt}
\renewcommand{\footrulewidth}{0.5pt}

\newcommand{\unc}[1]
{ \delta #1 }

\newcommand{\uncsq}[1]
{ \left(\unc{#1}\right)^2 }

\newcommand{\uncratio}[1]
{ \left(\frac{\unc{#1}}{#1}\right) }

\newcommand{\uncratiosq}[1]
{ \uncratio{#1}^2 }

\newcommand{\uncvector}[1]
{ \left[ \begin{array}{c} #1 \\ \delta #1 \end{array} \right] }

\newcommand{\comment}[1]
{{\bfseries \color{red} #1}}

\makeatletter
\renewcommand\section{\@startsection {section}{1}{\z@}%
                                   {-2.0ex \@plus -1ex \@minus -.2ex}%
                                   {2.3ex \@plus.2ex}%
                                   {\normalfont\bfseries}}% from \Large
\renewcommand\subsection{\@startsection{subsection}{2}{\z@}%
                                     {-2.0ex\@plus -1ex \@minus -.2ex}%
                                     {1.5ex \@plus .2ex}%
                                     {\normalfont\bfseries}}% from \large
\renewcommand\subsubsection{\@startsection{subsubsection}{3}{\z@}%
                                     {-2.0ex\@plus -1ex \@minus -.2ex}%
                                     {1.5ex \@plus .2ex}%
                                     {\normalfont\bfseries}}% from \normalsize
\makeatother


%----------------------------------------------------------------------------------------
%DOCUMENT INFORMATION
%----------------------------------------------------------------------------------------
\begin{document}
\begin{centering} Application Narrative for
    DE-FOA-XXXXXXXXXXX\\ \textbf{\large }\\ 20XX CFA Narrative xxxxxxx\\
\end{centering}
%\tableofcontents

\vspace{1em}

\noindent\textbf{Technical Workscope Identifier:}  \hspace{1.5in}
\textbf{Time Frame:}

\section{Verification and Validation in Nuclear Fuel Cycle Tools}

Nuclear fuel cycle simulation (NFCS) tools have a large scope of applications,
from the study of the behavior of a specific fuel type or reactor inside an
existing nuclear fleet to the prospective analysis of a complete transition to
advanced nuclear fuel cycle technologies. Unlike many simulation domains,
rigorous validation against experimental results is not viable for this kind of
simulation tool that incorporates aspects of human socio-political decision
making. Nevertheless, it is important to establish confidence in the quality of
the results that are generated, with different degrees of confidence depending
on the use case. The only existing way to develop confidence in any NFCS tool
is to compare with other similar tools or with simplified sets of historical
data. For the former, the conclusion of such a comparison is often a list of why
each software tool gives different results, but lacking quantitative information
on the precision of any of the results. The latter only allows validation of
existing concepts and has limited applicability to new fuel or reactor concepts
with varying technology readiness levels (TRLs).

There are a number of different tools used around the world to support decisions
on nuclear fuel cycle options. While some tools offer a rigorous treatment of
the physics, they often require an experienced analyst to compute the results.
The risk is to offer more modeling precisions than is warranted given the
uncertainty in the inputs and related socio-political constraints. It might lead
to an over-complicated simulation with a very complex analysis, and a false
sense of certainty in the results. Furthermore, tool development/design
philosophy can potentially influence simulation outcome of a fuel cycle
analysis.

Software verification and validation are two independent processes allowing
respectively to check that the software behaves as it is suppose to, and to
check that its responses it close to the reality, i.e. is it computing the right
thing versus is it computing the thing right.

While software verification can be achieve, using proper software testing
methods such as unit testing, validation in nuclear fuel cycle
simulation is almost impossible. Indeed, NFCS tools aims to simulate, the
historic or prospective, large nuclear fuel cycle of an entity (company, region,
country, \ldots). For obvious military and/or industrial strategic reasons most
of the data about existing nuclear fuel cycle is only partially publicly
available, if available.
Some NFCS tools, such as COSI developed by the CEA \cite{COSI6 - Coquelet}, have
been validated on real data but rigorous validation is limited in NFCS. Indeed
validation on historical data does not core a large spectrum of cases, as one of the main
application of the NFCS tools is prospective scenario simulation\ldots and is
generally not possible as those data are generally not publicly available.

Several inter-comparison studies between NFCS tools have been performed. Because
of their format and the wide bestiary of NFCS tools, those studies had a
relatively small impact on the NFCS community. Despite some very interesting
aspects and conclusions, those studies are generally comparison between large
and complex transition scenarios. Those complexities make it difficult to gain
deep understanding of the NFCS tools working processes. Moreover, those work are
usually lead by one NFCS tool development/using team, that drives the design of
the benchmark, and the other tool have to try to match the driver tool
philosophy. Those gaps in philosophy and their non accountability in the
benchmark design tend to limit the conclusion of such study. Additionally, some
example of such works can be found \cite{IAEA - Benchmark Study on Nuclear Fuel Cycle
Transition Scenarios} \cite{MIT - Guerin}, but generally either the definition
of the problem or its solutions generated by the different are not publicly
available. It is then impossible for other analysts to compare his own NFCS tool with
those studies\ldots

Some other interesting studies are the side product of simulation campaign such
as the Fuel Cycle Option (FCO) campaign. It such campaign multiple analysts,
using different NFCS tools, simulate the same transition scenario. The degree of
freedom in the transition specification varies from a campaign to an other,
which impacts the degree of agreement between the different calculations, but
those campaigns generally allow a deep comparison on specific problem between
several NFCS tools \cite{Standardized verification of fuel cycle modeling -
B.Feng}.  Even though those comparison are very valuable for the different tools
involved in the campaign the added values for outsider is very limited, as the
detailed specifications and results of those calculation are not completely
publicly available. Moreover those studies have a very narrow focus on few
specific transitions, intrinsically limiting the cross-comparison between the
NFCS tools. 


\section{Proposed objectives}

The aim of this project is to help the nuclear fuel cycle community to build
knowledge and understanding about the algorithms and models used by the
different fuel cycle simulation tools. Such knowledge could not only help fuel
cycle analysts to perform more pertinent and relevant study, but ultimately to
convince stakeholders to rely more on nuclear fuel cycle analysis in the
decision making processes relative to nuclear technology.


\section{Proposed scope description}

Verification and validation are to very important process in computer
simulation. As explain previously, validation can't be exactly achieve in NFCS
as the data required either not existent or not completely available. 

Nevertheless this project proposes open and collaborative solutions for NFCS
verification, and provide methods to build more confidence in NFCS as a
substitute for the validation.

The first part of this project will be dedicated to create a framework allowing
fuel cycle analysts to understand how specific modeling choices or tool designs
affect the simulation outcomes. This framework will include a variety of
problems defined to analyse a set of features and their range of implementations
in the different NFCS tools. It will allow a collaborative definition of
benchmark problems, the evaluation of simulation outputs for these problems, and
the cataloging of each participant’s contribution. 

The second part of this project will focus on linking modeling capability and
real study needs, aiming to assess the impact of modeling precision across
different facilities in the nuclear fuel cycle in order to determine the
robustness of the conclusions to those variations in modeling precision. 



\section{Logical path to accomplishing scope}

As previously explain this project is build around two complementary parts: the
first task will be dedicated to build a open collaborative benchmark framework
allowing real inter-code comparison and deep understanding of the implications
of the difference between NFCS tools. The second task will be focused on
answering the question: ``Which modeling fidelity is required to reach correct
conclusions regarding a specific fuel cycle scenario application ?''




\section{Logical path to accomplishing scope}


\subsection{FCCI}

This ultimate research goal is to establish a framework and
collaboration/comparison tools for defining comparison problems in a way that is
agnostic to the particular NFCS tool being testing and that facilitates
comparison of the results in a constructive and revealing manner.  This will be
demonstrated by defining an initial set of problems that any team can use to
test their own NFCS tool and gain a deeper understanding of the implications of
different development/modeling choices that are embedded in that tool. This will
be carried out in collaboration with the broader NFCS community. This will be
used as a showcase (through conferences contributions and scientific
publications) to promote the FCCI activities among the fuel cycle community. 


\subsubsection{Collaborative Framework for Problem Definition and Analysis} 

While it can be straightforward to define a problem for any single NFCS tool,
different modeling approaches and assumptions across a set of tools makes it
difficult to define those problems in a sufficiently generic manner.
Furthermore, the set of output data produced by each tool is different both in
the abstract (which data is generated) and the concrete (the format of that
data).  The success of an effort such as this lies in the ability for new
participants to easily access the problem definition and then also easily
compare their results to the previously generated and submitted results.

This task will focus on developing and formalizing the infrastructure needed for
effective collaboration, including a generalized ontology for defining canonical
problem inputs and outputs, mechanisms for proposing and refining new problem
definitions, submission of output data sets from a particular tool, and
web-based services to facilitate and automate the non-judgmental comparison of
the submitted outputs[3], all within an online community.  


\subsubsection{Problem definition} 

This task will be focused on the definition of new canonical problems that are
useful for demonstrating fundamental behavior of NFCS tools and/or highlighting
behaviors that distinguish different methodologies from each other.  A canonical
definition of the different problems and exercises should avoid any bias related
to the usage of a specific simulation tool and allow theoretically any tools
(with any conceptual design) to participate by simulating the same problem.  The
ontology developed above will help to accomplish this goal.  This will require
working with the FCCI collaboration to synthesize the different modeling
verification needs, and conceptualize problems that satisfy these needs while
allowing for deep analysis of potential differences due to modeling choices.
Experience from prior ad-hoc comparison efforts suggest that each problem’s
definition may require several updates based on the feedback of the different
analysts as their modeling progresses: reaching a canonical definition as an
asymptotic process. 


\subsubsection{Fuel Cycle Simulations and Analysis} 

This task will be dedicated to modeling the different problems of the FCCI, from
unit tests to the sensitivity analysis, using the Cyclus NFCS[2] developed at
the UW-Madison. The results of these simulations will be submitted, using the
infrastructure developed in Task 1, for analysis and comparison to submissions
generated using other NFCS tools.  These results will also, therefore, be
available to other participants as part of the comparison set for future
submissions.  The Cyclus framework includes multiple facility packages of
different fidelity levels, and the ability to swap individual facility models
within an otherwise constant fuel cycle scenario.  Thus, Cyclus can contribute
multiple submissions for each problem in the FCCI framework, each using a
different degree of fidelity.  Therefore, the FCCI framework will also serve as
a way to assess the impacts of these different fidelities with the Cyclus
ecosystem.  As new problems are identified and defined in Task 2, they will be
completed under this task.



\subsection{Robustness assessment}

The second part of this project will be dedicated to the
measurement/determination of the impact of different modeling choices on
different output metrics as a function of the studied scenario.  In the early
age of the nuclear fuel cycle simulation because of the limitation of the
computational power, the capabilities of the fuel cycle simulation were limited
and the subject to strong simplification. But, the refinement of the different
models did grows with the available computers power. 

The different actors of the fuel cycle community do not agree on the requirement
to use the highest modeling fidelity. While the primary intuition supports
always using the best modeling options, some experts argue that the different
hypothesis and unknown of a prospective scenario are largely dominate the
uncertainty on the output metrics; considering better modeling option is then a
waste of time and effort.

Nevertheless it is possible to pictures use cases where some higher level of
refinement are required to achieve correct or better conclusion. For example, a
simple description of the enrichment facility (a simple relation between the
different assay with a SWU constrain) will be enough for the simulation of the
historical a national fleet, a fuel cycle evaluation of the JCPOA agreement will
require a fine description of the enrichment cascade\dots Moreover there is no
scientific evaluation of the importance on the output metrics uncertainty of the
modeling precision over the input errors and uncertainties.  The part of the
project aims to evaluate the impact of the different modeling choices on the
output metrics and compare them to the one of the input metrics uncertainties.

The robustness part of the project will be divided into 3 phases, the experiment
definitions, the experiment building, and the analysis of the experiments.


\subsubsection{Experiment Definition}

As fuel cycle simulation can be used to support many different types of
decisions (facility siting, technology transitions, impacts of international
trade agreements, etc), a set of different nuclear fuel cycle scenarios will be
identified.

Multiple consultation meetings with the different stakeholders: politics,
industry, fco analyst and researchers. Those consultation will help to identify
the types of decisions that could be supported by different fuel cycle
calculation and the list of relevant output metrics for each applications. 

Some examples include: availability/reduction of resources (thorium vs uranium
vs plutonium), reduction of long term waste radio-toxicity, and minimization of
human manipulation of activated material.

In addition to the scenarios identified with those consultations, other nuclear
fuel cycle applications will be considered, some examples includes: US (large
heterogeneous fleet) and/or French (fleet undergoing transition to plutonium
recycling) historical fleet, EG-like transition as well as, non-proliferation
application such as Iran case.

In this compilation of use scenarios, about five will be identified as ``most
representative`` of the different nuclear fuel cycle usages and used as
''reference scenarios`` for the rest of this study. In addition to those
''reference scenarios'', set of output metrics will be identified for each use
cases, with an associated acceptable tolerances, within decisions/conclusions
can be made.


\subsubsection{The experiments}

This second task will be be split into two subtasks dedicated to robustness
assessment, the first one to variations in modeling choices, the second one to
input variations.  

Each identified scenario will be modeled with a range of existing Cyclus
facility models.  Previously funded efforts have resulted in a variety of
reactor models: fleet-based recipe reactors [4], individual recipe reactors [5],
and depletion-aware/capable reactors [6, 7, 8].  As the reactors become more
sophisticated, it may also be necessary to invoke more sophisticated models for
fuel fabrication that can respond to varying reactor fueling needs [7, 8]. The
Cyclus framework makes it possible to shuffle the various modeling options, for
example using a single depletion aware reactor with any fuel fabrication model,
to explore the impact of a single modeling change while the rest of the
simulation remains unchanged.  

In addition to probing the impact of modeling choice, this work will also seek
to understand how robust a given decision is to variations in the input
assumptions for a given scenario.  A second round of analyses will be performed
to establish the robustness of the output metrics to variation on the fuel cycle
parameters such as cooling/reprocessing time, enrichment tails, facility
throughput, as well as transition schedule (start date, speed\dots).

Combined, those analyses will provide a way to rank different modeling choices
and input variables by importance depending on the outcome of the fuel cycle
simulations.


\subsubsection{Experiment Analysis}

The final task will be dedicated to the analysis and the compilation of the
required modeling fidelity as well as the input parameters constrains required
to successfully support the different decision making processes.

Those analysis will provide guidance not only to the nuclear fuel cycle analysts
regarding the necessary level of model fidelity that is required to support
particular kinds of decision making, but also to fuel cycle researcher regarding
the improvement needed on the differents modeling capabilities.


\section{Schedule}




\section{Milestones \& Deliverables}



\section{Quality Assurance \& Software Development Process}

\textit{This project will comply with all Quality Assurance requirements, as
described on the NEUP website.}

This work will follow the Cyclus development model.  We will employ a variety of
modern software project management tools: revision control, automated testing,
automated and manual documentation, bug tracking. Related projects at are
already managed under source code revision control system (git) that provides
detailed tracking of all changes to the code base. A test suite has been
developed and new tests will be added for each new capability. This test suite
will be automatically executed with each proposed change in the revision control
system to identify any flaws that may be introduced. Automated documentation
tools are used in the source code to create a detailed reference for the
interfaces and additional background documentation. Out-of-code detailed reports
and publications will supplement the information on each new capability.
Finally, a bug tracking system (GitHub issues) is deployed to help users and
developers to understand known issues and to track their resolution as a
developer community. As new capability is added, it will come under the same
quality assurance practices as described here for the existing capabilities.

%----------------------------------------------------------------------------------------
%BIBLIOGRAPHY
%----------------------------------------------------------------------------------------

\bibliographystyle{narrative}

\bibliography{narrative}

%----------------------------------------------------------------------------------------

\label{LastPage} \end{document}
