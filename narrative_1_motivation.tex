\section{Verification and Validation in Nuclear Fuel Cycle Tools}

Nuclear fuel cycle simulation (NFCS) tools have a large scope of applications,
from the study of the behavior of a specific fuel type or reactor inside an
existing nuclear fleet to the prospective analysis of a complete transition to
advanced nuclear fuel cycle technologies. Unlike many simulation domains,
rigorous validation against experimental results is not viable for this kind of
simulation tool that incorporates aspects of human socio-political decision
making. Nevertheless, it is important to establish confidence in the quality of
the results that are generated, with different degrees of confidence depending
on the use case. The only existing way to develop confidence in any NFCS tool
is to compare with other similar tools or with simplified sets of historical
data. For the former, the conclusion of such a comparison is often a list of why
each software tool gives different results, but lacking quantitative information
on the precision of any of the results. The latter only allows validation of
existing concepts and has limited applicability to new fuel or reactor concepts
with varying technology readiness levels (TRLs).

There are a number of different tools used around the world to support decisions
on nuclear fuel cycle options. While some tools offer a rigorous treatment of
the physics, they often require an experienced analyst to compute the results.
The risk is to offer more modeling precisions than is warranted given the
uncertainty in the inputs and related socio-political constraints. It might lead
to an over-complicated simulation with a very complex analysis, and a false
sense of certainty in the results. Furthermore, tool development/design
philosophy can potentially influence simulation outcome of a fuel cycle
analysis.

Software verification and validation are two independent processes that
check that the software behaves as it is suppose to, and 
check that its responses it close to the reality, respectively.  That is,
has the model been implemented correctly \emph{versus} has the correct
model been implemented.

While software verification can be achieve, using proper software testing
methods such as unit testing, validation in nuclear fuel cycle
simulation is almost impossible. Indeed, NFCS tools aims to simulate, the
historic or prospective, large nuclear fuel cycle of an entity (company, region,
country, \ldots). For security and/or industrial strategic reasons most
of the data about existing nuclear fuel cycle is only partially publicly
available, if available.
Some NFCS tools, such as COSI developed by the CEA \cite{COSI6 - Coquelet}, have
been validated on real data but rigorous validation is limited in NFCS. Indeed
validation on historical data does not core a large spectrum of cases, as one of the main
application of the NFCS tools is prospective scenario simulation\ldots and is
generally not possible as those data are generally not publicly available.
\todo{This paragraph needs a better finish.}

Several inter-comparison studies between NFCS tools have been performed. Because
of their format and the wide variety of NFCS tools, those studies had a
relatively small impact on the NFCS community, despite some very interesting
features and conclusions.  Those studies are generally comparison between large
and complex transition scenarios whose complexities make it difficult to gain
deep understanding of the NFCS tools working processes. Moreover, those studies are
often led by one NFCS tool development/using team, driving the design of
the benchmark such that other tools have to try to match the driver tool's
philosophy. Those gaps in philosophy and their non accountability in the
\todo{What do you mean by ``their non-accountability''?}
benchmark design tend to limit the conclusion of such study. Additionally, some
example of such works can be found \cite{IAEA - Benchmark Study on Nuclear Fuel Cycle
Transition Scenarios} \cite{MIT - Guerin}, but generally either the definition
of the problem or its solutions generated by the different are not publicly
available. It is then overly challenging for analysts not involved in the original
study to compare their own NFCS tool with those results.\ldots

Although not formally constructed as benchmark exercises, some simulation campaigns
can also produce results for comparison purposes.  In the US Department of Energy's
Fuel Cycle Option (FCO) campaign,  multiple analysts
using different NFCS tools simulate the same transition scenario. The degree of
freedom in the transition specification varies from one scenario to an other,
which impacts the degree of agreement between the different calculations, but
those campaigns generally allow a deep comparison on specific problem between
several NFCS tools \cite{Standardized verification of fuel cycle modeling -
B.Feng}.  Even though such comparisons are very valuable for the different tools
involved in the campaign, the added value for outsiders is very limited, as the
detailed specifications and results of those calculations are not always
publicly available. Moreover these studies have a very narrow focus on a few
specific transitions, intrinsically limiting the cross-comparison between the
NFCS tools.\todo{why does this limit the cross-comparison?}


\section{Proposed objectives}

Verification and validation are to very important process in computer
simulation. As explain previously, validation can't be exactly achieve in NFCS
as the data required either not existent or not completely available.
Nevertheless this project proposes open and collaborative solutions for NFCS
verification, and provide methods to build more confidence in NFCS as a
substitute for the validation.
The aim of this project is to help the nuclear fuel cycle community to build
knowledge and understanding about the algorithms and models used by the
different fuel cycle simulation tools. Such knowledge could not only help fuel
cycle analysts to perform more pertinent and relevant study, but ultimately to
convince stakeholders to rely more on nuclear fuel cycle analysis in the
decision making processes relative to nuclear technology.



\section{Proposed scope description}


The first part of this project will be dedicated to create a framework allowing
fuel cycle analysts to understand how specific modeling choices or tool designs
affect the simulation outcomes. This framework will include a variety of
problems defined to analyse a set of features and their range of implementations
in the different NFCS tools. It will allow a collaborative definition of
benchmark problems, the evaluation of simulation outputs for these problems, and
the cataloging of each participant’s contribution.

The second part of this project will focus on linking modeling capability and
real study needs, aiming to assess the impact of modeling precision across
different facilities in the nuclear fuel cycle in order to determine the
robustness of the conclusions to those variations in modeling precision.



\section{Logical path to accomplishing scope}

As previously explain this project is build around two complementary parts: the
first task will be dedicated to build a open collaborative benchmark framework
allowing real inter-code comparison and deep understanding of the implications
of the difference between NFCS tools. The second task will be focused on
answering the question: ``Which modeling fidelity is required to reach correct
conclusions regarding a specific fuel cycle scenario application ?''


